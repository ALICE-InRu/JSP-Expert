% TEXINPUTS=.:/home/hei2/Documents/LaTeX/myTexStyles:
% created for Journal of Heuristics 
% Time-stamp: "2011-07-18 14:17 hei2"
% if all fonts computer modern use -G1
% dvips -Ppdf -G0 <filename>
% http://www.springer.de/comp/lncs/authors.html

%!TEX TS-program = xelatex
%!TEX encoding = UTF-8 Unicode

%! xelatex joh.tex

\RequirePackage{fix-cm} 

\documentclass[smallextended]{svjour3} 
\smartqed  % flush right qed marks, e.g. at end of proof

\makeatletter
%\def\cl@chapter{\cl@chapter \@elt {theorem}}%bug in class
\def\cl@chapter{\@elt {theorem}}
\makeatother

\usepackage{mathptmx} % use Times fonts if available on your TeX system

\journalname{Journal of Heuristics}

\title{Learning Linear Composite Dispatch Rules for Scheduling}
\subtitle{Case study for the job- and flow-shop problem}

\author{Helga Ingimundardottir \and Thomas Philip Runarsson }
%\authorrunning{Short form of author list} % if too long for running head

\institute{H. Ingimundardottir \at
	Dunhaga 5, IS-107 Reykjavik, Iceland \\
	Tel.: +354-525-4704\\
	Fax: +354-525-4632\\
	\email{hei2@hi.is}\\
	\and
	T.P. Runarsson \at
	Hjardarhagi 2-6, IS-107 Reykjavik, Iceland \\
	Tel.: +354-525-4733\\
	Fax: +354-525-4632\\
	\email{tpr@hi.is}\\
}
\date{Received: \today / Accepted: date}
% The correct dates will be entered by the editor


\usepackage[english]{babel} 

% please place your own definitions here and don't use \def but
% \newcommand{}{}
\usepackage{url}

\usepackage{amssymb,bm,amsmath}
\newcommand{\vphi}{\bm{\phi}}
\newcommand{\vsigma}{\bm \sigma}
\newcommand{\vchi}{\bm \chi}
\newcommand{\R}{{\mathbb R}}

\newcommand{\inner}[2]{\big<{#1}\cdot{#2}\big>}
\newcommand{\abs}[1]{\lvert#1\rvert}
\newcommand{\norm}[1]{\lVert#1\rVert}
\newcommand{\argmax}{\mathop{\rm argmax}}
\newcommand{\nchoosek}[2]{\tiny \left(\begin{array}{c}#1\\#2\end{array}\right)}
\newcommand{\condset}[2]{\left\{#1\;\middle|\;#2\right\}}

% percentage relative deviation from optimality
\newcommand{\Namerho}{Deviation from optimality, $\rho$}
\newcommand{\namerho}{deviation from optimality, $\rho$}
\newcommand{\fullnamerho}{\namerho, defined by~\cref{eq:rho}}
\newcommand{\Problem}[2][ ]{$\mathcal{P}_{#2}^{#1}$}
\newcommand{\jrnd}[2]{\Problem[#1 \times #2]{j.rnd}}
\newcommand{\jrndJ}[2]{\Problem[#1 \times #2]{j.rnd,J_1}}
\newcommand{\jrndM}[2]{\Problem[#1 \times #2]{j.rnd,M_1}}
\newcommand{\jrndn}[2]{\Problem[#1 \times #2]{j.rndn}}
\newcommand{\frnd}[2]{\Problem[#1 \times #2]{f.rnd}}
\newcommand{\frndn}[2]{\Problem[#1 \times #2]{f.rndn}}
\newcommand{\fjc}[2]{\Problem[#1 \times #2]{f.jc}}
\newcommand{\fmc}[2]{\Problem[#1 \times #2]{f.mc}}
\newcommand{\fmxc}[2]{\Problem[#1 \times #2]{f.mxc}}
\newcommand{\dr}{dispatching rule}
\newcommand{\cdr}{composite priority \dr}
\newcommand{\sdr}{single priority \dr}
\newcommand{\Fsp}{Flow-shop}
\newcommand{\Jsp}{Job-shop}
\newcommand{\fsp}{flow-shop}
\newcommand{\jsp}{job-shop}
\newcommand{\FSP}{FSP}
\newcommand{\JSP}{JSP}
\newcommand{\Jrnd}{\JSP~random}
\newcommand{\JrndJ}{\JSP~random with job variation}
\newcommand{\JrndM}{\JSP~random with machine variation}
\newcommand{\Jrndn}{\JSP~random-narrow}
\newcommand{\Frnd}{\FSP~random}
\newcommand{\Frndn}{\FSP~random narrow}
\newcommand{\Fjc}{\FSP~job-correlated}
\newcommand{\Fmc}{\FSP~machine-correlated}
\newcommand{\Fmxc}{\FSP~mixed-correlated}

\newcounter{FeatCounter}
% job-related
\newcommand{\phiproc}{$\phi_1$}
\newcommand{\phistartTime}{$\phi_2$}
\newcommand{\phiendTime}{$\phi_3$}
\newcommand{\phiarrivalTime}{$\phi_4$}
\newcommand{\phitotalProc}{$\phi_5$}
\newcommand{\phiwait}{$\phi_6$}
\newcommand{\phiwrmJob}{$\phi_7$}
\newcommand{\phijobOps}{$\phi_8$}
\newcommand{\phiJobRelated}{\phiproc\!-\phijobOps}
% mac-related
\newcommand{\phimacFree}{$\phi_{9}$}
\newcommand{\phiwrmMac}{$\phi_{10}$}
\newcommand{\phimacOps}{$\phi_{11}$}
\newcommand{\phiMacRelated}{\phimacFree\!-\phimacOps}
% flow-related
\newcommand{\phislotsReduced}{$\phi_{12}$} 
\newcommand{\phislots}{$\phi_{13}$}
\newcommand{\phislotsTotal}{$\phi_{14}$} 
\newcommand{\phiFlowRelated}{\phislotsReduced\!-\phislotsTotal}
% schedule related
\newcommand{\phimakespan}{$\phi_{15}$}
\newcommand{\phiScheduleRelated}{\phimakespan}
% global
\newcommand{\phiSPT}{$\phi_{16}$}
\newcommand{\phiLPT}{$\phi_{17}$}
\newcommand{\phiLWR}{$\phi_{18}$}
\newcommand{\phiMWR}{$\phi_{19}$}
\newcommand{\phiRND}{$\phi_{20}$}
\newcommand{\phiSDRRelated}{\phiSPT\!-\phiMWR}
\newcommand{\phiGlobalRelated}{\phiMWR\!-\phiRND}
\newcommand{\phiLocalRelated}{\phiproc\!-\phimakespan}
\newcommand{\NrFeatLocal}{15}
\newcommand{\NrFeatGlobal}{5}
\newcommand{\NrFeatTotal}{20}

\hyphenation{heur-ist-ics}
\hyphenation{algo-rithm}

\usepackage{multirow}
\usepackage{rotating}
\newcommand{\rot}[1]{\begin{sideways}#1\end{sideways}}

\usepackage{paralist}

\usepackage{booktabs} % \toprule \midrule \bottomrule

\usepackage[colorinlistoftodos, textwidth=\marginparwidth]{todonotes}
\usepackage{comment}

\usepackage[capitalise,nameinlink]{cleveref} % must come last! % put your own shorthand declarations in this document
\input{../shorthandCommon}
\usepackage{amssymb,bm,amsmath}
\renewcommand{\vphi}{\bm \phi}
\renewcommand{\vsigma}{\bm \sigma}
\renewcommand{\vchi}{\bm \chi}


\begin{document}
\maketitle
	
\selectlanguage{english}
	
\begin{abstract}
  % \todo{Try for one sentence or so on}
		
  % \todo[inline]{Motivation}
  Instead of creating new dispatching rules in an ad-hoc manner,
  % \todo[inline]{Method}
  this study gives a framework on how to study simple heuristics for
  scheduling problems.  Before starting to create new composite
  dispatching rules, meticulous research on optimal schedules can give
  an abundance of valuable information that can be utilised for
  learning new models.  For instance, it's possible to seek out when
  the scheduling process is most susceptible to failure.  Furthermore,
  the stepwise optimality of individual features imply their
  explanatory predictability. From which, a preference set is
  collected and a preference based learning model is created based on
  what feature states are preferable to others w.r.t. the end result,
  here minimising the final makespan.
  % \todo[inline]{Key result}
  By doing so it's possible to learn new composite dispatching rules
  that outperform the models they are based on.
  % \todo[inline]{Conclusion}
  Even though this study is based around the job-shop scheduling
  problem, it can be generalised to any kind of combinatorial problem.
		
  \keywords{Scheduling \and Composite dispatching rules \and Machine Learning \and Feature Selection}
\end{abstract}
	
	
% ----------------------------- Introduce idea
\section{Introduction}\label{sec:introduction}
\todo[inline]{Lure the reader in a with a good first sentence (which
  this is not!)}
	
\todo{What is the problem?} A subclass of scheduling problems is the
job-shop scheduling problem (JSP), which is widely studied in
operations research.  Job-shop deals with the allocation of tasks of
competing resources where its goal is to optimise a single or multiple
objectives.  Its analogy is from manufacturing industry where a set of
jobs are broken down into tasks that must be processed on several
machines in a workshop.  \todo{Why is it interesting?} Furthermore,
its formulation can be applied on a wide variety of practical problems
in real-life applications which involve decision making, therefore its
problem-solving capabilities has a high impact on many manufacturing
organisations.
	
JSP is NP-hard \cite{Garey76:NPhard}, hence finding optimal solutions
of high dimensionality is exceedingly difficult in a reasonable amount
of time. As a result heuristics methods are adopted. Generally, this
is done by applying a hand-crafted dispatching rule (DR) for a given
problem space. Due to the exorbitant amounts of DRs to choose from,
and any kind of alteration to the problem space, this can be come
quite a time-consuming selection process for the heuristic designer,
which any kind of automation would alleviate immensely. \todo{What are
  your contributions?}  For this reason, we propose a framework for
learning the indicators of optimal solutions, such as done by
\cite{Siggi10}.  The study shows that during the scheduling process it
varies \emph{when} it's most fruitful to make the `right' decision,
and depending on the problem space those pivotal moments can vary
greatly. \todo{What is the outline of what you will show?}  Although,
using optimal trajectory for creating training data gives vital
information on how to learn good scheduling rules, it is a good
starting point, but not sufficient. This is due to the fact our models
are only based on optimal decisions, then once we make a suboptimal
choice we are in uncharted territory and its effects are relatively
unknown. For this reason, it is of paramount importance to inspect the
actual end-performance when choosing a suitable model, not just
staring blindly at the training accuracy. Moreover, different measures
on how to report training accuracy is discussed.
	
The outline of the paper is the following, \cref{sec:problemdef} gives
the mathematical formalities of the scheduling problem, and
\cref{sec:constructionjssp} goes into how their schedules are
constructed, followed by \cref{ch:learningmodels} giving a background
on what has been done previously in learning new dispatching rules in
similar fields. \Cref{sec:opt} sets up the framework for learning from
optimal schedules. In particular, the probability of choosing optimal
decisions and the effects of making a suboptimal
decision. Furthermore, the optimality of common dispatching rules is
investigated, from which a blended dispatching rule is created. With
those guidelines, \cref{ch:expr:CDR} goes into detail how to create
meaningful composite dispatching rules, with the importance of good
feature selection and the polysemy of how to report training
accuracy. The paper finally concludes in \cref{sec:con} with
discussion and conclusions.
	
% \section{Background}
% \todo[inline]{Often need to set scene} \todo[inline]{Define
% formalism} \todo[inline]{Get reader up to speed}
% \todo[inline]{Identify research problem}
		
\section{Job and Permutation Flow-Shop
  Scheduling}\label{sec:problemdef}
% JSSP-samantekt:
The job-shop problem (JSP) involves the scheduling of jobs on a set of
machines. Each job consists of a number of operations which are then
processed on the machines in a predetermined order. An optimal
solution to the problem will depend on the specific objective.
% For example, the optimal schedule may be the time needed to complete
% all jobs, i.e., the minimum makespan.
	
In this study we will consider the $n\times m$ JSP, where $n$ jobs,
$\mathcal{J}=\{J_j\}_{j=1}^n$, are scheduled on a finite set,
$\mathcal{M}=\{M_a\}_{a=1}^m$, of $m$ machines. The jobs are subject
to the constraint that each job $J_j$ must follow a predefined machine
order, a chain of $m$ operations
$\vsigma_j=\{\sigma_{j1},\sigma_{j2},\dotsc,\sigma_{jm}\}$. Furthermore,
a machine can handle at most one job at a time. Additional constraints
commonly considered are job release-dates and due-dates, however,
those will not be considered here.  The objective will be to schedule
the jobs so as to minimize the maximum completion times for all tasks,
also known as the makespan, $C_{\max}$. A common notion for this
family of scheduling problems is $J||C_{\max}$ \citep{Pinedo08}.  In the
case when all jobs share the same permutation route $\vsigma_j$, the
JSP is reduced to a permutation flow-shop scheduling problem (FSP)
\citep{Guinet1998,Tay08}, denoted $F||C_{\max}$. Therefore, without
the loss of generality, this study will be structured around the JSP.
		
Henceforth the index $j$ refers to a job $J_j\in\mathcal{J}$ while the
index $a$ refers to a machine $M_a\in\mathcal{M}$. If a job requires a
number of processing steps or operations, then the pair $(j,a)$ refers
to the operation, i.e., processing the task of job $J_j$ on machine
$M_a$. Note that once an operation is started, it must be completed
uninterrupted, i.e., pre-emption is not allowed. Moreover, there are
no sequence dependent setup times.
		
	
	

\section{Scheduling Heuristics} \label{sec:constructionjssp}
	
Heuristics algorithms for scheduling are typically either a
construction or improvement heuristics. The improvement heuristic
starts with a complete schedule and then tries to find similar but
better schedules.  A construction heuristic starts with an empty
schedule and adds one job at a time until the schedule is complete.
The work presented here will focus on construction heuristics,
although the techniques developed could be adapted to improvement
heuristics also. In scheduling a construction heuristic is typically
implemented as a priority dispatching rule. These are simple rules
that basically determine which incomplete job should be dispatched
next. However, knowing which job to dispatch is not sufficient, one
must also know where to place it. In order to build tight schedules it
is sensible to place a job, once it becomes available, such that
the machine idle time is minimal. There may also be a number of
different options for such a placement. \Cref{fig:jssp:example}
illustrates the dispatching process with an example of a temporal
partial schedule of six jobs scheduled on five-machines. The numbers
in the boxes represent the job identification $j$. The width of the
box illustrates the processing times for a given job for a particular
machine $M_a$ (on the vertical axis). The dashed boxes represent the
resulting partial schedule for when a particular job is scheduled
next. Moreover, the current $C_{\max}$ is denoted by a dotted vertical
line. In the figure we observe that job 2, to be scheduled on machine
3, could be placed immediately in a slot between job 3 and 4, or after
job 4. If job 6 had been placed earlier a slot would
have been created between it and job 4 thus creating a third
alternative (job 2 after job 6). The construction heuristic must
therefore decide where to place the job, this may be independent of
the dispatching rule applied. Different placement strategies could be
considered, for example placing a job in smallest feasible slot. In
our experiments we have discovered that such a placement can
potentially rule out the possibility of constructing optimal
schedules. This problem, however, did not occur when jobs are simply
placed as early as possible. For this reason it will be our placement strategy.
	
\begin{figure}[t!]\centering
  \includegraphics[width=0.8\textwidth]{jssp_example_nocolor-eps-converted-to.pdf}
  \caption[Gantt chart of a partial JSP schedule]{Gantt chart of a
    partial JSP schedule after 15 dispatches: Solid and dashed boxes
    represent $\vchi$ and $\mathcal{L}^{(16)}$, respectively. Current
    $C_{\max}$ denoted as dotted line.}
  \label{fig:jssp:example}
\end{figure}
	
	
The sequential ordering of jobs dispatched to machines, i.e. $(j,a)$; the collective set of
allocated tasks to machines, form what we will refer to as a \emph{sequence}. A \emph{scheduling policy} will
pertain to the manner in which the sequence is determined.  As shown
in our example given in \Cref{fig:jssp:example}, there are $15$
operations already scheduled. The sequence used to create the schedule
was,
\begin{eqnarray}
  \vchi=\left(J_3,J_3,J_3,J_3,J_4,J_4,J_5,J_1,J_1,J_2,J_4,J_6,J_4,J_5,J_3\right)
\end{eqnarray}
and the available jobs to be scheduled
$\mathcal{L}^{(k)}=\{J_1,J_2,J_4,J_5,J_6\}$ describes the five potential jobs to be dispatched
at step $k=16$ (note that job 3 is completed). An overview on dispatching rules, used to create such sequences, is given below.
	
\subsection{Priority Dispatching Rules}
	
% \subsection{Simple priority dispatching
% rules}\label{ch:dispatchrules}
%Dispatching rules are of a construction heuristics, where one starts with an empty schedule and adds sequentially on %one operation (or tasks) at a time. 
	
A priority dispatching rule  inspects the job-list
 and dispatches the job with the highest priority. These
rules typically use attributes for the corresponding operation, for example the 
processing time for the job. Consider again \Cref{fig:jssp:example}, if the
job with the shortest processing time (SPT) were to be scheduled next
then $J_2$ would be dispatched. Similarly, for the longest processing
time (LPT) heuristic $J_5$ would be dispatched. Dispatching can
also be based on attributes related to the partial schedule. Examples of these are 
dispatching the job with the most work remaining (MWR) or alternatively the least work remaining
(LWR). A survey of more than
$100$ of such rules are presented in \citet{Panwalkar77}, however the
reader is referred to an in-depth survey for single-priority or
\emph{simple dispatching rules} (SDR) by \citet{Haupt89}.  SDRs assign
an index to each job in the job-list and
is generally only based on few attributes and simple mathematical
operations.

\begin{table}[t!] \centering
  \caption[Attribute space $\mathcal{A}$ for JPS]{Attribute space $\mathcal{A}$ for JSP where job $J_j$ on machine $M_a$ given the resulting temporal schedule after dispatching $(j,a)$.
  }
  \label{tbl:jssp:feat}
  \centering
\renewcommand{\arraystretch}{1.5}
\begin{tabular}{clll} %p{0.45\textwidth}|p{0.4\textwidth}|}
	\toprule
	$\vphi$          & Feature description                       & Mathematical formulation                                                           & Shorthand    \\ 
	%\hline  \multicolumn{4}{c}{\textbf{Local features}}  \\
	\midrule
	\multicolumn{4}{c}{\textbf{job related}}\\
	\phiproc         & job processing time                       & $p_{ja}$                                                                           & proc         \\
	\phistartTime    & job start-time                            & $x_s(j,a)$                                                                         & startTime    \\
	\phiendTime      & job end-time                              & 
	$x_e(j,a)$                                                                    
	     &
	 endTime      \\
	\phiarrivalTime  & job arrival time                          & 
	$x_e(j,a-1)$                                                                  
	     &
	 arrival      \\ 
	\phitotalProc    & total processing time                     & $\sum_{a\in \mathcal{M}}p_{ja}$                                                    & totalProc    \\
	\phiwait         & time job had to wait                      & 
	$x_s(j,a)-x_e(j,a-1) 
	$                                                             & wait         
	\\   
	\phiwrmJob       & total work remaining for job              & $\sum_{a'\in\mathcal{M}\setminus \mathcal{M}_{j}}p_{ja'}$                          & wrmJob       \\
	\phijobOps       & number of assigned operations for job     & $|\mathcal{M}_j|$                                                                  & jobOps       \\ 
	\midrule
	\multicolumn{4}{c}{\textbf{machine related}}\\
	\phimacFree      & when machine is next free                 & $\max_{j'\in 
	\mathcal{J}_a} \{x_e(j',a)\}$                                         & 
	macFree      \\
	\phiwrmMac       & total work remaining for machine          & $\sum_{j'\in\mathcal{J}\setminus \mathcal{J}_{a}}p_{j'a} $                         & wrmMac       \\
	\phimacOps       & number of assigned operations for machine & $|\mathcal{J}_a|$                                                                  & macOps       \\
	\phislotsReduced & change in idle time by assignment         & $\Delta 
	s(a,j)$                                                                    & 
	reducedSlack \\
	\phislots        & total idle time for machine               & $\sum_{j'\in 
	\mathcal{J}_a}s(a,j')$                                                & 
	slack        \\
	\phislotsTotal   & total idle time for all machines          & $\sum_{a'\in 
	\mathcal{M}}\sum_{j'\in \mathcal{J}_{a'}}s(a',j')$                    & 
	totalSlack   \\
	\phimakespan     & current makespan                          & 
	$\max_{(j',a')\in \mathcal{J} \times 
	\mathcal{M}_{j'}}\{x_f(j',a')\}$              & makespan     \\
	\bottomrule
\end{tabular}


\end{table}
	
Designing priority dispatching rules requires recognizing the
important attributes of the partial schedules needed to create a good
scheduling rule. These attributes attempt to grasp key features of the schedule being
constructed. Which attributes are most important will necessarily depend on the
objectives of the scheduling problem. 
 Attributes used in this study applied for a job $J_j$ to
be dispatched on machine $M_a$ are given in \cref{tbl:jssp:feat}.
The attributes of particular interest were obtained by inspecting the
aforementioned SDRs from \cref{ch:dispatchrules}. Attributes \phiJobRelated\ and
\phiMacRelated\ are job-related and machine-related, respectively.
Then there are flow-related attributes, \phiFlowRelated\, which measure the influence of idle
time on the schedule, and current makespan related, \phiScheduleRelated.
All of these attributes vary throughout the scheduling process,
w.r.t. operation belonging to the same time step $k$, with the exception of
\phimac, which is reported in order to distinguish which features are
in conflict with each other;
({\bf Helga: not sure how this works again... check code, can you recheck if this is all OK here? I mean do we need to talk about things we don't use in our experimental study!}
 \phistep\ to keep track of features'
evolution w.r.t. the scheduling process; and \phitotalProc\ and
\phiwrmTotal\ which are static for a given problem instance, but used
for normalising other features, e.g. \phiwrmTotal\ for work-remaining
based ones (\phiwrmJob\ and \phiwrmMac).

Dispatching rules are attractive
since they are relatively easy to implement, fast and find good
schedules. However, they can also fail unpredictably. Combining
different SDRs can potentially enhance the
scheduling performance. 
	
	
\subsection{Composite Priority Dispatching Rules}\label{sec:CDR}
A careful combination of dispatching rules can perform significantly
better \cite{Jayamohan04}. These are referred to as \emph{composite
  dispatching rules} (CDR), where the priority ranking is an
expression of several single-based priority dispatching rules. CDRs
can deal with greater number of features and more complicated form, in
short, CDR are a combination of several SDRs. For instance let CDR be
comprised of $d$ dispatching rules (DR), then the index $I$ for job $J_j$ using CDR is,
\begin{equation}
  I_j^{CDR} = \sum_{i=1}^d w_i \cdot \text{DR}_i(\vphi_j) \label{eq:CDR}
\end{equation}
where $w_i>0$ and $\sum_{i=0}^d w_i = 1$ and $w_i$ gives the weight of
the influence of $\text{DR}_i$ (which could be SDR or another CDR) to
CDR. Note, each $\text{DR}_i$ is function of the job $J_j$'s attributes
 $\vphi_j$.  \nomenclature[zdr2]{CDR}{composite priority
  dispatching rule} \nomenclature[zdr3]{BDR}{blended dispatching rule}
	
Since each DR yield a priority index $I^{DR}$ then it is easy to
translate its index as a performance measure $a$. Then it is possible
to combine several performance measures into a single DR, these are
referred to as blended dispatching rules (BDR), where an overall
blended priority index $P$ is defined as
\begin{equation}
  P_j = \sum_{a=l} w_a \cdot a 
\end{equation}
where $w_a>0$ and $\sum_{a=0}^C w_a = 1$ and $w_a$ gives the weight of
the proportional influence of performance measure $a$ (based on some
SDR or CDR) to the overall priority.

{\bf Helga can you please make it clear what the difference is between a blended and composite rule, for me its seems to be the same thing... are we confusing things here?!}

At each time step $k$, an operation is dispatched which has the
highest priority in the job-list,
$\mathcal{L}^{(k)}\subset\mathcal{J}$.  If there is a tie, some other priority measure
is used. Generally the priority dispatching rules are static during
the entire scheduling process.
	

{\bf Helga: reword this paragraph so you don't start the sentence with a citation}
\citet{Lu13} investigate 11 simple dispatching rules for JSP to create
a pool of 33 composite dispatching rules that strongly outperformed
the ones they were based on, which is intuitive since where one SDR
might be failing, another could be excelling so combining them
together should yield a better CDR. \citet{Lu13} create their
composite dispatching rules with multi-contextual functions (MCFs)
based on either on machine idle time or job waiting time, so one can
say that the composite dispatching rules are a combination of those
two key features of the schedule and then the basic dispatching
rules. However, there are no combinations of the basic DR explored,
only machine idle time and job waiting time.  \citet{Yu13} used
priority rules to combine 12 existing dispatching rules from the
literature, in their approach they had 48 priority rules combinations,
yielding 48 different models to implement and test. This is a fairly
ad-hoc solution and there is no guarantee the optimal combination of
dispatching rules is found.

	
Generally the weights $\vec{w}$ are chosen by the designer or the rule
apriori.  A more attractive approach would be to learn these weights from problem examples directly. We will now investigate how this may be accomplished.

\section{Learning Dispatching Rules}\label{ch:learningmodels}
	

	
A recent editorial of the state-of-the-art approaches in advanced
dispatching rules for large-scale manufacturing systems by
\citet{Chen13} points out that:
\lq\lq ... most traditional dispatching rules are based on historical
  data. With the emergence of data mining and online analytic
  processing, dispatching rules can now take predictive information
  into account\rq\rq. The importance automated discovery of DR was also emphasised by \cite{Monch13}. 
Several of successful implementations in the field of
semiconductor wafer fabrication facilities are discussed, however, this sort of
investigation is still in its infancy.

{\bf Helga: the remainder of this chapter should be about how learning has been used to find compostite dispatching rules, from the literature, here you can cite you own work and the work of Olafson and thos citing him. This section should conclude with a paragraph on instance generation and training data creation to connect to the nect chapter. I leave this here below in case you would like to use something from it ...}

With meta heuristics one can use existing DRs and use for example
portfolio-based algorithm selection \citep{Rice76,Gomes01}, either
based on a single instance or class of instances \citep{Xu07} to
determine which DR to choose from. 
	
\citet{Kalyanakrishnan11} point out that meta learning can be very
fruitful in reinforcement learning, and in their experiments they
discovered some key discriminants between competing algorithms for
their particular problem instances, which provided them with a hybrid
algorithm which combines the strengths of the algorithms.
	
\citet{Nguyen13} proposed a novel iterative dispatching rules (IDRs)
for JSP which learns from completed schedules in order to iteratively
improve new ones. At each dispatching step, the method can utilise the
current feature space to \emph{correctify} some possible \emph{bad}
dispatch made previously (sort of reverse lookahead).  Their method is
straightforward, and thus easy to implement and more importantly
computationally inexpensive, although the authors do stress that there
is still remains room for improvement.
	
\citet{Korytkowski13} implement ant colony optimisation to select the
best DR from a selection of nine DRs for JSP and their experiments
showed that the choice of DR do affect the results and that for all
performance measures considered it was better to have a all the DRs to
choose from rather than just a single DR at a time.
	

\section{Learning from Problem Instances}\label{sec:gentrainingdata}

{\bf HELGA: what is this  chapter is about?} This chapter need a rewrite, I will let you take the first iteration.

\subsection{Problem Instances}


{\bf Helga: put here all material releated to \cref{tbl:data:sim}. }

For each problem class described in \cref{tbl:data:sim} there are $N$
problem instances generated with a random problem generator using $n$
jobs and $m$ machines.  The goal is to minimize the makespan,
$C_{\max}$. The optimum makespan is denoted $C_{\max}^{\text{opt}}$,
and the makespan obtained from the scheduling policy $A$ under
inspection by $C_{\max}^{A}$. Since the optimal makespan varies
between problem instances the performance measure is the following,
\begin{equation}\label{eq:ratio}\rho=\frac{C_{\max}^{A}-C_{\max}^{\text{opt}}}{C_{\max}^{\text{opt}}}\cdot
  100\%\end{equation}
	which indicates the percentage relative deviation from optimality. %Note, for the OR-Library benchmark suite the optimum is not known, in those instances $C_{\max}^{opt}$ is swapped for $C_{\max}^{BKS}$ which is the latest best known solution reported in the literature. 
	\nomenclature[so]{opt}{(known) optimum}
	\nomenclature[ss]{sub}{sub-optimum}
	\nomenclature[zdr0]{DR}{dispatching rule}
\nomenclature[osizepruned]{$N$}{number of problem instances}
	
\subsection{Schedule building}\label{sec:gen:gametree}
When building a complete schedule $\ell=n\cdot m$ dispatches must be
made sequentially.  A job is placed at the earliest available time
slot for its next machine, whilst still fulfilling that each machine
can handle at most one job at each time, and jobs need to have
finished their previous machines according to its machine order.
Unfinished jobs are dispatched one at a time according to some
heuristic. After each dispatch\footnote{Dispatch and time step are
  used interchangeably.} the schedule's current features
(cf. \cref{tbl:jssp:feat}) are updated based on the half-finished
schedule.
	
It is easy to see that the sequence of task assignments is by no means
unique. Inspecting a partial schedule further along in the dispatching
process such as in \cref{fig:jssp:example}, then let's say $J_1$ would
be dispatched next, and in the next iteration $J_2$. Now this sequence
would yield the same schedule as if $J_2$ would have been dispatched
first and then $J_1$ in the next iteration, i.e. these are
non-conflicting jobs.  In this particular instance one can not infer
that choosing $J_1$ is better and $J_2$ is worse (or vice versa) since
they can both yield the same solution.
	
Note that in some cases there can be multiple optimal solutions to the
same problem instance. Hence not only is the sequence representation
`flawed' in the sense that slight permutations on the sequence are in
fact equivalent w.r.t. the end-result, but very varying permutations
on the dispatching sequence (however given the same partial initial
sequence) can result in very different complete schedules but can
still achieve the same makespan, and thus same deviation from
optimality, $\rho$, defined by \eqref{eq:ratio}, which is the measure
under consideration. Care must be taken in this case that neither
resulting features are labelled as undesirable. Only the resulting
features from a dispatch resulting in a suboptimal solution should be
labelled undesirable.
	
	\subsection{Labelling schedules w.r.t. optimal decisions}
	The optimum makespan is known for each problem instance. 
	At each time step a number of feature pair are created, they consist of the features $\vphi_o$ resulting from optimal dispatches $o\in\mathcal{O}^{(k)}$, versus features $\vphi_s$ resulting from suboptimal dispatches $s\in\mathcal{S}^{(k)}$ at time $k$. Note, $\mathcal{O}^{(k)}\cup\mathcal{S}^{(k)}=\mathcal{L}^{(k)}$ and $\mathcal{O}^{(k)}\cap\mathcal{S}^{(k)}=\emptyset$.
	In particular, each job is compared against another job of the job-list, $\mathcal{L}^{(k)}$, and if the makespan differs, i.e. $C_{\max}^{(s)}\gneq C_{\max}^{(o)}$, an optimal/suboptimal pair is created, however if the makespan would be unaltered the pair is omitted since they give the same optimal makespan. This way, only features from a dispatch resulting in a suboptimal solution is labelled undesirable.
	
	The approach taken here is to verify analytically, at each time step, by fixing the current temporal schedule as an initial state, whether it can indeed \emph{somehow} yield an optimal schedule by manipulating the remainder of the sequence. This also takes care of the scenario that having dispatched a job resulting in a different temporal makespan would have resulted in the same final makespan if another optimal dispatching sequence would have been chosen. That is to say the data generation takes into consideration when there are multiple optimal solutions to the same problem instance. 
	
	
	
	\subsection{Creating time-independent dispatching rules}\label{sec:ord:timeindependent}
	
	Preliminary experiments for creating step-by-step model was done in \cite{InRu11a} where an optimal trajectory was explored, i.e. at each dispatch some (random) optimal task is dispatched, resulting in local linear model for each dispatch; a total of $\ell$ linear models for solving $n\times m$ JSP. However, the experiments there showed that by fixing the weights to its mean value throughout the dispatching sequence, results remained satisfactory.  
	A more sophisticated way, would be to create a \emph{new} linear model, where the preference set, $S$, is the union of the preference pairs across the $\ell$ dispatches. This would amount to a substantial training set, and for $S$ to be computationally feasible to learn, $S$ has to be reduced. For this several ranking strategies were explored in \cite{InRu14b}, the results there showed that it's sufficient to use partial subsequent rankings, namely, combinations of $r_i$ and $r_{i+1}$ for $i\in\{1,\ldots,n'\}$, are added to the training set, where $r_1>r_2>\ldots>r_{n'}$ ($n'\leq n$) are the rankings of the job-list, $\mathcal{L}^{(k)}$, at time step $k$, in such a manner that in the cases that there are more than one operation with the same ranking, only one of that rank is needed to be compared to the subsequent rank. Moreover, in the case of this study, which deals with $10\times 10$ problem instances, the partial subsequent ranking becomes necessary, as full ranking is computationally infeasible. This is due to the since the size of the training set, $\abs{S}$, becomes too large with full ranking, and would need sampling.
	
	
	\subsection{Linear Learning}
	
	{\bf Helga: this is a condensed version of liblinear for our problem, please do not describe logistic regression just how the data is preprocessed and fed into liblinear.}
	
	Learning models considered in this study are based on ordinal regression in which the learning task is formulated as learning preferences. In the case of scheduling, learning which operations are preferred to others. Ordinal regression has been previously presented in \cite{Ru06:PPSN} and in \cite{InRu11a} for JSP, however given here for completeness. 
	
	Let $\vphi_{o}\in\R^d$ denote the post-decision state when dispatching $J_o$ corresponds to an optimal schedule being built. All post-decisions states corresponding to suboptimal dispatches, $J_s$, are denoted by $\vphi_{s}\in\R^d$. One could label which feature sets were considered optimal, $\vec{z}_{o}=\vphi_{o}-\vphi_{s}$, and suboptimal, $\vec{z}_{s}=\vphi_{s}-\vphi_{o}$ by $y_o=+1$ and $y_s=-1$ respectively. 
	Note, a negative example is only created as long as $J_s$ actually results in a worse makespan, i.e. $C_{\max}^{(s)}\gneq C_{\max}^{(o)}$, since there can exist situations in which more than one operation can be considered optimal.
	
	The preference learning problem is specified by a set of preference pairs,
	\begin{equation}
	S = \left\{\left\{\vec{z}_o,+1\right\}_{k=1}^{\ell},\left\{\vec{z}_s,-1\right\}_{k=1}^{\ell}
	\;|\;\forall o\in \mathcal{O}^{(k)},s\in \mathcal{S}^{(k)}
	\right\}\subset \Phi\times Y \label{eq:Sjssp}
	\end{equation}
	where $\Phi\subset \mathbb{R}^d$ is the training set of $d$ features, 
	$Y=\{-1,+1\}$ is the outcome space, $\ell=n\times m$ is the total number dispatches, 
	from which $o\in\mathcal{O}^{(k)}$ and $s\in \mathcal{S}^{(k)}$ denote optimal and suboptimal dispatches, respectively, at step $k$. 
	Note, $\mathcal{O}^{(k)}\cup\mathcal{S}^{(k)}=\mathcal{L}^{(k)}$, and $\mathcal{O}^{(k)}\cap\mathcal{S}^{(k)}=\emptyset$. 
	
	For JSP there are $d=\NrFeatLocal$ features (cf. \cref{tbl:jssp:feat} and explained in more detail in \cref{sec:CDR}), and the training set is created in the manner described in \cref{sec:gentrainingdata}.
	\nomenclature[oe]{$\ell$}{number of dispatches needed for a complete schedule, $\ell=n\cdot m$}
	\nomenclature[sk]{$k$}{refers to dispatch/time step $k$ for a schedule}
	\nomenclature[ophi]{$\vphi_k$}{feature set, i.e. post-decision state, of a (partial) schedule at time $k$}
	\nomenclature[od]{$d$}{number of distinct features, i.e. dimension of $\vphi$}
	\nomenclature[ophi]{$\Phi$}{training set}
	\nomenclature[ophi2]{$S$}{preference set}
	\nomenclature[oset1]{$\mathcal{O}^{(k)}$}{set of optimal dispatches at time  $k$}
	\nomenclature[oset2]{$\mathcal{S}^{(k)}$}{set of suboptimal dispatches at time  $k$}
	
	Now consider the model space $\mathcal{H} = \{h(\cdot) : X \mapsto Y\}$ of mappings from solutions to ranks. Each such function $h$ induces an ordering $\succ$ on the solutions  by the following rule,
	\begin{equation}\label{eq:linear}
	\vec{x}_i \succ \vec{x}_j \quad \Leftrightarrow \quad h(\vec{x}_i) > h(\vec{x}_j)
	\end{equation}
	where the symbol $\succ$ denotes "is preferred to".  The function used to induce the preference is defined by a linear function in the feature space,
	\begin{equation}\label{eq:jssp:linweights}
	h(\vec{x})=\sum_{i=1}^d w_i\phi_i(\vec{x})=\inner{\vec{w}}{\vphi(\vec{x})}.
	\end{equation}
	Let $\vec{z}$ denote either $\vphi_o-\vphi_s$ with $y=+1$ or $\vphi_s-\vphi_0$ with $y=-1$ (positive and negative example respectively). Logistic regression learns the optimal parameters $\vec{w}\in\mathbb{R}^d$ determined by solving the following task, 
	\begin{equation}\label{eq:margin}
	\min_{\vec{w}}\quad \tfrac{1}{2}\inner{\vec{w}}{\vec{w}} + C \sum_{i=1}^{\abs{S}} \log\left(1 + e^{-y_i \inner{\vec{w}}{\vec{z}_i}}\right) 
	\end{equation}
	where $C > 0$ is a penalty parameter, and the negative log-likelihood is due to the fact the given data point $\vec{z}_i$ and weights $\vec{w}$ are assumed to follow the probability model,
	\begin{equation}\label{eq:prob}
	\Prob{y=\pm1|\vec{z},\vec{w}}=\frac{1}{1+e^{-y\inner{\vec{w}}{\vec{z}_i}}}.
	\end{equation}
	The logistic regression defined in \eqref{eq:margin} is solved iteratively, in particular using Trust Region Newton method \cite{Lin08:newtontrustregion}, which generates a sequence $\{\vec{w}^{(k)}\}_{k=1}^\infty$ converging to the optimal solution $\vec{w}^*$ of \eqref{eq:margin}.
	
	The regulation parameter $C$ in \eqref{eq:margin}, controls the balance between model complexity and training errors, and must be chosen appropriately. It is also important to scale the features $\vphi$ first. A standard method of doing so is by scaling the training set such that all points are in some range, typically $[-1,1]$. That is, scaled $\tilde{\vphi}$ is,
	\begin{equation}\label{eq:scale}
	\tilde{\phi}_i = 2 (\phi_i - \underline{\phi}_i) / (\overline{\phi}_i - \underline{\phi}_i) - 1 
	\quad\quad \forall i\in\{1,\ldots,d\}
	\end{equation}
	where $\underline{\phi}_i$, $\overline{\phi}_i$ are the maximum and minimum $i$-th component of all the feature variables in set $\Phi$, namely,
	\begin{equation}
	\underline{\phi}_i=\min\{\phi_i\;|\;\forall\vphi\in \Phi\} \quad\textrm{and}\quad \overline{\phi}_i=\max\{\phi_i\;|\;\forall\vphi\in \Phi\}.
	\end{equation}
	where $i\in\{1\ldots d\}$. Moreover, scaling makes the features less sensitive to processing times.
	
	Logistic regression makes optimal decisions regarding optimal dispatches and at the same time efficiently estimates a posteriori probabilities. The optimal $\vec{w}^*$ obtained by the training set, can be used on any new data point, $\vphi$, and their inner product is proportional to probability estimate \eqref{eq:prob}. Hence, for each job on the job-list, $J_j\in\mathcal{L}$, let $\vphi_j$ denote its corresponding  post-decision state. Then the job chosen to be dispatched, $J_{j^*}$, is the one corresponding to the highest preference estimate, i.e.,
	\begin{equation}\label{eq:lin}
	J_{j^*}=\argmax_{J_j\in \mathcal{L}}\; h(\vphi_j)
	\end{equation}
	where $h(\cdot)$ is the classification model obtained by the preference set, $S$, defined by \eqref{eq:Sjssp}. 
	
	\subsection{Interpreting linear classification models}\label{sec:learningmodels:interpret}
	Looking at the features description in \cref{tbl:jssp:feat} it is possible for the ordinal regression to `discover' the weights $\vec{w}$ in order for \eqref{eq:jssp:linweights} corresponds to applying a single priority dispatching rules from \cref{ch:dispatchrules}. For instance, 
	\begin{eqnarray*} %{s'rCl}
	SPT:~~w_i&=&\bigg\{ \begin{array}{rl}-1&\text{if }i=1\\0&\text{otherwise}\end{array} \\
	LPT:~~w_i&=&\bigg\{ \begin{array}{rl}+1&\text{if }i=1\\0&\text{otherwise}\end{array} \\
	MWR:~~w_i&=&\bigg\{ \begin{array}{rl}+1&\text{if }i=7\\0&\text{otherwise}\end{array} \\
	LWR:~~w_i&=&\bigg\{ \begin{array}{rl}-1&\text{if }i=7\\0&\text{otherwise}\end{array}
	\end{eqnarray*}
	where $i\in\{1,\ldots,d\}$. % Note, that at each time step $k$ a task corresponding to the \emph{highest} priority is chosen, i.e. $\argmax_{J_j\in\mathcal{L}}\{h(\vphi_j)\}$.
	When using a feature space based on SDRs, the linear classification models can very easily be interpreted as CDRs with predetermined weights.
	
	
	%----------------------------- Develop idea
	
	%\subsubsection{Problem spaces}\label{sec:datadescription}
	For this study synthetic JSP and PFSP problem instances will be considered with the problem sizes $8\times8$,  $10\times10$ and $12\times12$. Summary of problem classes is given in \cref{tbl:data:sim}.
	Note, that difficult problem instances are not filtered out beforehand, such as the approach in \citet{Whitley}. 
	
	%\subsubsection{Job-shop}\label{data:sim:jssp}
	Problem instances for JSP are generated stochastically by fixing the number of jobs and machines and 
	discrete processing time are i.i.d. and sampled from a discrete uniform distribution from the interval $I=[u_1,u_2]$, i.e. $\vec{p}\sim \mathcal{U}(u_1,u_2)$. 
	Two different processing times distributions were explored, namely 
	\jrnd{n}{m}  where $I=[1,99]$ and \jrndn{n}{m}  where $I=[45,55]$.
	The machine order is a random permutation of all of the machines in the job-shop, hence they problem spaces \jrnd{n}{m}   and \jrndn{n}{m}  are referred to as random and random-narrow, respectively. 
	
	For each JSP class $N_{\text{train}}$  and $N_{\text{test}}$ instances were generated for training and testing, respectively. Values for $N$ are given in \cref{tbl:data:sim}. 
	
	Although in the case of \jrnd{n}{m}  this may be an excessively large range for the uniform distribution, it is however chosen in accordance with the literature \citep{Demirkol98} for creating synthesised $J||C_{\max}$ problem instances. In addition, w.r.t. the machine ordering, one could look into a subset of JSP where the machines are partitioned into two (or more) sets, where all jobs must be processed on the machines from the first set (in some random order) before being processed on any machine in the second set, commonly denoted as $J|2\textrm{sets}|C_{\max}$ problems, but as discussed in \cite{orlib_swv} this family of JSP is considered "hard" (w.r.t. relative error from best known solution) in comparison with the "easy" or "unchallenging" family with the general $J||C_{\max}$ setup. % ath. Holtsclaw96 vitnar í orlib_swv um easy-hard pælinguna
	This is in stark contrast to \citet{Whitley} whose findings showed that structured $F||C_{\max}$ were quite easier to solve than completely random structures. 
	Intuitively, an inherent structure in machine ordering should be exploitable for a better performance.  However, for the sake of generality, a random structure is preferred as they correspond to difficult problem instances in the case of JSP. 
	%Whereas, structured problem subclasses will be explored for PFSP.  
	
	%\subsubsection{Flow-shop}\label{data:sim:fsp}
	Problem instances for PFSP are such that processing times are i.i.d. and uniformly distributed, 
	\frnd{n}{m}  where $\vec{p}\sim\mathcal{U}(1,99)$, referred to as random. In the JSP context \frnd{n}{m}  is analogous to \jrnd{n}{m}.
	
	There are $N_{\text{train}}$  and $N_{\text{test}}$ instances were generated for training and testing, respectively. Values for $N$ are given in \cref{tbl:data:sim}. 
	
	\begin{table}\centering
	\caption[Problem space distributions used in experimental studies.]{Problem space distributions used in experimental studies. Note, problem instances are synthetic and each problem space is i.i.d. and `--' denotes not available.}\label{tbl:data:sim}
	{\renewcommand{\arraystretch}{1.5}
	\begin{tabular}{clcccl}\toprule 
		type & name & size ($n\times m$) & $N_{\text{train}}$ & 
		$N_{\text{test}}$ & note
		\\ \midrule
		\multirow{2}{*}{{JSP}}
		%&\jrnd{8}{8} &$8\times8$& -- & 500 & random \\
		  & \jrnd{10}{10}  & $10\times10$ & 300 & 200 & random        \\
		%&\jrnd{12}{12} &$12\times12$& -- & 500 & random \\
		%&\jrndn{8}{8} &$8\times8$& -- & 500 & random-narrow \\ 
		  & \jrndn{10}{10} & $10\times10$ & 300 & 200 & random-narrow \\ 
		%&\jrndn{12}{12} &$12\times12$& -- & 500 & random-narrow \\ 
		\midrule
		\multirow{1}{*}{{FSP}}
		%&\frnd{8}{8} &$8\times8$& --&500& random \\ 
		  & \frnd{10}{10}  & $10\times10$ & 300 & 200 & random        \\ 
		%&\frnd{12}{12} &$12\times12$& --&500& random \\ 
		\bottomrule
	\end{tabular}
}
	\end{table}
	
	
	
	
	\section{Performance of SDR and BDR}\label{sec:opt}
	In order to create successful dispatching rules, a good starting point is to investigate the properties of optimal solutions and hopefully be able to learn how to mimic such "good" behaviour. For this, we follow an optimal solution, obtained by using a commercial software package \cite{gurobi}, and inspect the evolution of its features, defined in \cref{tbl:jssp:feat}. Moreover, it is noted, that there are several optimal solutions available for each problem instance. However, it is deemed sufficient to inspect only one optimal trajectory per problem instance as there are $N_{\text{train}}=300$ independent instances which gives the training data variety. 
	
	%Note, for this \lcnamecref{sec:opt}, only $10\times10$ problem instances will be considered from the problem spaces described in \cref{tbl:data:sim}. Leaving dimensionality $8\times8$ and $12\times12$ solely for testing scalability in \cref{sec:scalability}. 
	%Note, figures within this \lcnamecref{sec:opt} depict the mean over all the training data. %, which are quite noisy functions. Thus, for clarity purposes, they are fitted with local polynomial regression, making the boundary points biased.
	
	\subsection{Probability of choosing optimal decision}\label{sec:opt:rnd}
	Firstly, we can observe that on a step-by-step basis there are several optimal dispatches to choose from. \Cref{fig:opt:unique} depicts how the number of optimal dispatches evolve at each dispatch iteration. Note, that only one optimal trajectory is pursued (chosen at random), hence this is only a lower bound of uniqueness of optimal solutions.
	As the number of possible dispatches decrease over time, \cref{fig:opt} depicts the probability of choosing an optimal dispatch. 
	
	\begin{figure}
		\centering
		\includegraphics[width=1\linewidth]{figures/{trdat.prob.optUniqueness.10x10.OPT}.pdf}
		\caption{Number of unique optimal dispatches (lower bound)}
		\label{fig:opt:unique}
	\end{figure}
	
	\begin{figure}
		\centering
		\includegraphics[width=1\linewidth]{figures/{trdat.prob.moveIsOptimal.10x10.OPT.matlab}.pdf}
		\caption{Probability of choosing optimal move}
		\label{fig:opt}
	\end{figure}
	
	\subsection{Making suboptimal decisions}\label{sec:opt:sub}
	Looking at \cref{fig:opt}, \jrnd{10}{10}  has a relatively high probability ($70\%$ and above) of choosing an optimal job. However, it is imperative to keep making optimal decisions, because once off the optimal track the consequences can be dire. To demonstrate this \cref{fig:case} depicts mean worst and best case scenario of the resulting deviation from optimality, $\rho$, once you've fallen off the optimal track. Note, that this is given that you make \emph{one} wrong turn. Generally, there will be more, and then the compound effects of making suboptimal decisions really start adding up. 
	
	It is interesting that for JSP, that over time making suboptimal decisions make more of an impact on the resulting makespan. This is most likely due to the fact that if suboptimal decision is made in the early stages, then there is space to rectify the situation with the subsequent dispatches. However, if done at a later point in time, little is to be done as the damage is already inflicted upon the schedule. However, for FSP, the case is the exact opposite. Then it's imperative to make good decisions right from the beginning. This is due to the major structural differences between JSP and FSP, namely the latter having a homogeneous machine ordering, constricting the solution immensely. Luckily, this does have the added benefit of making it less vulnerable for suboptimal decisions later in the decision process. 
	
	
	\begin{figure}
		\centering
		\includegraphics[width=1\linewidth]{figures/{stepwise.10x10.OPT.casescenario}.pdf}
		\caption{Mean deviation from optimality, $\rho$, (\%), for best (lower bound) and worst (upper bound) case scenario of choosing suboptimal dispatch for \jrnd{10}{10}, \jrndn{10}{10} and \frnd{10}{10}}
		\label{fig:case}
	\end{figure}
	
	\subsection{Optimality of simple priority dispatching rules}\label{sec:opt:sdr}
	The probability of optimality of the aforementioned SDRs from \cref{ch:dispatchrules}, yet still maintaining our optimal trajectory, i.e., the probability of a job chosen by a SDR being able to yield an optimal makespan on a step-by-step basis, is depicted  in   \cref{fig:opt:SDR}. Moreover, the dashed line represents the benchmark of random guessing (cf. \cref{fig:opt}).
	
	Now, let's bare in mind the deviation from optimality of applying SDRs throughout the dispatching process (box-plots of which are depicted in \cref{fig:boxplot:SDR}) then there is a some correspondence between high probability of stepwise optimality and low $\rho$. Alas, this isn't always the case, for \jrnd{10}{10}, SPT always outperforms LPT w.r.t. stepwise optimality, however this does not transcend to SPT having a lower $\rho$ value than LPT. Hence, it's not enough to just learn optimal behaviour, one needs to investigate what happens once we encounter suboptimal state spaces.
	
	\begin{figure}
		\centering
		\includegraphics[width=1\linewidth]{figures/{trdat.prob.moveIsOptimal.10x10.SDR.matlab}.pdf}
		\caption{Probability of SDR being optimal}
		\label{fig:opt:SDR}
	\end{figure}
	
	\begin{figure}
		\centering
		\includegraphics[width=1\linewidth]{figures/{boxplotRho.SDR.10x10}.pdf}
		\caption{Box plot for deviation from optimality, $\rho$, (\%) for SDRs}
		\label{fig:boxplot:SDR}
	\end{figure}
	
	\subsection{Simple blended dispatching rule}\label{sec:opt:bdr}
	A naive approach to create a simple blended dispatching rule would be for instance be switching between two SDRs at a predetermined time point. Hence, going back to \cref{fig:opt:SDR} a presumably good BDR for \jrnd{10}{10}  would be starting with SPT and then switching over to MWR at around time step 40, where the SDRs change places in outperforming one another. A box-plot for $\rho$ for all problem spaces is depicted in \cref{fig:boxplot:BDR}. Now, this little manipulation between SDRs does outperform SPT immensely, yet doesn't manage to gain the performance edge of MWR, save for \frnd{10}{10}. This gives us insight that for job-shop based problem spaces, the attribute based on MWR is quite fruitful for good dispatches, whereas the same cannot be said about SPT -- a more sophisticated BDR is needed to improve upon MWR. 
	
	A reason for this lack of performance of our proposed BDR is perhaps that by starting out with SPT in the beginning, it sets up the schedules in such a way that it's quite greedy and only takes into consideration jobs with shortest immediate processing times. Now, even though it is possible to find optimal schedules from this scenario, as \cref{fig:opt:SDR} show, the inherent structure that's already taking place, and might make it hard to come across by simple methods. Therefore it's by no means guaranteed that by simply swapping over to MWR will handle that situation which applying SPT has already created. \Cref{fig:boxplot:BDR} does however show, that by applying MWR instead of SPT in the latter stages, does help the schedule to be more compact w.r.t. SPT. However, in the case of \jrnd{10}{10}  and \jrndn{10}{10}  the fact remains that the schedules have diverged too far from what MWR would have been able to achieve on its own. Preferably the blended dispatching rule should use  best of both worlds, and outperform all of its inherited DRs, otherwise it goes without saying one would simply still use the original DR that achieved the best results.
	
	\begin{figure}
		\centering
		\includegraphics[width=1\linewidth]{figures/{boxplotRho.BDR.10x10}.pdf}
		\caption{Box plot for deviation from optimality, $\rho$, (\%) for BDR where SPT is applied for the first 40\% of the dispatches, followed by MWR}
		\label{fig:boxplot:BDR}
	\end{figure}
	
	\begin{comment}
	\subsection{Extremal feature}\label{sec:opt:ext}
	The SDRs we've inspected so-far are based on two features from \cref{tbl:jssp:feat}, namely
	\begin{itemize}
	\item \phiproc\ for SPT and LPT 
	\item \phiwrmJob\ for LWR and MWR 
	\end{itemize}
	by choosing the lowest value for the first SDR, and highest value for the latter SDR, i.e., the extremal values for those given features. Let's apply the same methodology from \cref{sec:opt:sdr} to all varying features\footnote{Note, \phistep, \phimac\ and \phiwrmTotal\ describe the features, not the schedule. For instance, \phistep\, gives us no new information, as that feature is homogeneous for each timestep, making it equivalent to random guessing.} described in \cref{tbl:jssp:feat}.  \Cref{fig:j.rnd:opt:minmax,fig:j.rndn:opt:minmax,fig:f.rnd:opt:minmax}
	depict the probability of all extremal features being an optimal dispatch, with random guessing from \cref{fig:opt} as a dashed line. 
	
	In order to put the extremal features into perspective, it's worth comparing them with how the evolution of the features are over time, depicted in \cref{fig:j.rnd:opt:evol,fig:j.rndn:opt:evol,fig:f.rnd:opt:evol}. 
	
	
	\begin{figure}
	\centering
	\includegraphics[width=1\linewidth]{figures/{j.rnd}/{trdat.feat.stepwise.10x10.OPT}.pdf}
	\caption{Feature evolution of optimal trajectory for \jrnd{10}{10}}
	\label{fig:j.rnd:opt:evol}
	\end{figure}
	\begin{figure}
	\centering
	\missingfigure{j.rndn}
	%\includegraphics[width=1\linewidth]{figures/{j.rndn}/{trdat.feat.stepwise.10x10.OPT}.pdf}
	\caption{Feature evolution of optimal trajectory for \jrndn{10}{10}}
	\label{fig:j.rndn:opt:evol}
	\end{figure}
	\begin{figure}
	\centering
	\includegraphics[width=1\linewidth]{figures/{f.rnd}/{trdat.feat.stepwise.10x10.OPT}.pdf}
	\caption{Feature evolution of optimal trajectory for \frnd{10}{10}}
	\label{fig:f.rnd:opt:evol}
	\end{figure}
	
	\begin{figure}
	\centering
	\includegraphics[width=1\linewidth]{figures/{j.rnd}/{trdat.prob.moveIsOptimal.10x10.feat.minmax}.pdf}
	\caption{Probability of extremal feature being optimal for \jrnd{10}{10}}
	\label{fig:j.rnd:opt:minmax}
	\end{figure}
	\begin{figure}
	\centering
	\includegraphics[width=1\linewidth]{figures/{j.rndn}/{trdat.prob.moveIsOptimal.10x10.feat.minmax}.pdf}
	\caption{Probability of extremal feature being optimal for \jrndn{10}{10}}
	\label{fig:j.rndn:opt:minmax}
	\end{figure}
	
	\begin{figure}
	\centering
	\includegraphics[width=1\linewidth]{figures/{f.rnd}/{trdat.prob.moveIsOptimal.10x10.feat.minmax}.pdf}
	\caption{Probability of extremal feature being optimal for \frnd{10}{10}}
	\label{fig:f.rnd:opt:minmax}
	\end{figure}
	\end{comment}
	
	\section{Learning CDR}\label{ch:expr:CDR}
	\Cref{sec:opt:bdr} demonstrates there is definitely something to be gained by trying out different combinations, it's just non-trivial how to go about it, and motivates how it's best to go about learning such interaction, which will be addressed in this \lcnamecref{ch:expr:CDR}.
	
	\subsection{Feature Selection}
	The SDRs we've inspected so-far are based on two features from \cref{tbl:jssp:feat}, namely
	\begin{itemize}
		\item \phiproc\ for SPT and LPT 
		\item \phiwrmJob\ for LWR and MWR 
	\end{itemize}
	by choosing the lowest value for the first SDR, and highest value for the latter SDR, i.e., the extremal values for those given features. 
	There is nothing that limits us to using just those two features. 
	From \cref{tbl:jssp:feat} we will limit our experiments to the first $d=16$ features, as they are varying for each operation, save for \phitotalProc\ which is varying for each $J_j\in\mathcal{J}$. 
	
	For this study we will consider all combinations of features using either one, two, three or all of the features, for a total of $\nchoosek{d}{1}+\nchoosek{d}{2}+\nchoosek{d}{3}+\nchoosek{d}{d}$, i.e., total of 697 combinations. The reason for such a limiting number of active features, are due to the fact we want to keep the models simple enough for improved model interpretability
	
	For each feature combination, a linear preference model is created in the manner described in \cref{ch:learningmodels}, where $\Phi$ is limited to the predetermined feature combination. This was done with the software package from \cite{liblinear}\footnote{Software available at \url{http://www.csie.ntu.edu.tw/~cjlin/liblinear}}, by training on the full preference set $S$ obtained from the $N_{\text{train}}=300$ problem instances following the framework set up in \cref{sec:gentrainingdata}. 
	
	\subsection{Training accuracy}\label{sec:CDR:acc}
	As the preference set $S$ has both preference pairs belonging to optimal ranking, and subsequent rankings, it is not of primary importance to classify \emph{all} rankings correctly, just the optimal ones. Therefore, instead of reporting the training accuracy based on the classification problem of the correctly labelling the problem set $S$, it's opted the training accuracy is obtained in the same manner as done in \cref{sec:opt:sdr} for SDRs, i.e., the probability of choosing optimal decision given the resulting linear weights, however in this context, the mean throughout the dispatching process is reported. \Cref{fig:stepwise_vs_classification} shows the difference between the two measures of reporting training accuracy. Training accuracy based on stepwise optimality only takes into consideration the likelihood of choosing the optimal move at each time step. However, the classification accuracy is also trying to correctly distinguish all subsequent rankings in addition of choosing the optimal move, as expected that measure is considerably lower. 
	
	\begin{figure}[th!]
		\centering
		\includegraphics[width=\linewidth]{figures/exhaust/{training.accuracy.equal}.pdf}
		\caption{Various methods of reporting training accuracy for preference learning}
		\label{fig:stepwise_vs_classification}
	\end{figure}
	
	\subsection{Pareto front}\label{sec:CDR:pareto}
	When training the learning model one wants to keep the training accuracy high, as that would imply a higher likelihood of making optimal decisions, which would in turn translate into a low final makespan. To test the validity of this assumptions, each of the 697 models is run on the preference set, and its mean $\rho$ is reported against its corresponding training accuracy in \cref{fig:CDR:scatter}. The models are colour-coded w.r.t. the number of active features, and a line is drawn through its Pareto front. Moreover, those solutions are labelled with their corresponding model ID. Moreover, the Pareto front over all 697 models, irrespective of active feature count, is denoted with triangles. Moreover, their values are reported in \cref{tbl:CDR:pareto}, where the best objective is given in boldface. 
	Note for \jrndn{10}{10}  there is no statistical difference between models 3.501, 3.508 and 3.510 w.r.t. training accuracy, however only 3.501 and 3.508 w.r.t. $\rho$. Other models were statistically significant to one another, using a Kolmogorov-Smirnov test with $\alpha=0.05$.\label{sec:expr:ks}
	
	Note, for both \jrnd{10}{10} and \jrndn{10}{10}, model 1.16 is on the Pareto front. The model corresponds to feature \phiwrmJob, and in both cases has a weight strictly greater than zero (cf. \cref{fig:CDR:weights}). Revisiting \cref{sec:learningmodels:interpret}, we observe that this implies the learning model was able to discover MWR as one of the Pareto solutions. 
	
	As one can see from \cref{fig:CDR:scatter}, adding additional features to express the linear model boosts performance in both training accuracy and expected mean for $\rho$, i.e., the Pareto fronts are cascading towards more desirable outcome with higher number  of active features. However, there is a cut-off point for such improvement, as using all features is generally considerably worse off. 
	
	\begin{figure}[t]
		\centering
		\includegraphics[width=\linewidth]{figures/exhaust/{pareto.equal}.pdf}
		\caption{Scatter plot for training accuracy  (\%) against its corresponding mean expected $\rho$ (\%) for all 697 linear models, based on either one, two, three or all $d$ combinations of features.
			Pareto fronts for each active feature count based on maximum training accuracy and minimum mean expected $\rho$ (\%), and labelled with their model ID. Moreover, actual Pareto front over all models is marked with triangles.} \label{fig:CDR:scatter}
	\end{figure}
	
	\begin{table}
		\caption{Mean training accuracy and mean expected deviation from optimality, $\rho$, for all CDR models on the Pareto front from \cref{fig:CDR:scatter}.}\label{tbl:CDR:pareto}
		\centering
\begin{tabular}{cr@{.}lllc}\toprule
Problem & NrFeat & Model & Acc & $\rho$ & Pareto \\ 
\midrule \multirow{10}{*}{\jrnd} 
   & 1 & 3 & 90.03 & 34.08 &  \\ 
   & 1 & 4 & 74.74 & 21.41 &  \\ 
   & 1 & 16 & 85.43 & 22.06 &  \\ 
   & 2 & 94 & 91.34 & 32.84 &  \\ 
   & 2 & 108 & 91.10 & 13.32 &  \\ 
   & 2 & 115 & 91.08 & 13.31 &  \\ 
   & 3 & 382 & 91.94 & 47.53 &  \\ 
   & 3 & 473 & 91.86 & 23.31 & $\blacktriangle$ \\ 
   & 3 & 549 & 91.68 & \textbf{13.26} & $\blacktriangle$ \\ 
   & 16 & 1 & \textbf{91.95} & 33.96 & $\blacktriangle$ \\ 
\midrule \multirow{11}{*}{\jrndn} 
   & 1 & 4 & 75.38 & 18.84 &  \\ 
   & 1 & 15 & 85.26 & 46.77 &  \\ 
   & 1 & 16 & 84.72 & 19.66 &  \\ 
   & 2 & 113 & 88.53 & 19.66 &  \\ 
   & 2 & 116 & 87.04 & 13.52 &  \\ 
   & 3 & 274 & \textbf{89.82} & 31.39 & $\blacktriangle$ \\ 
   & 3 & 499 & 89.68 & 15.19 & $\blacktriangle$ \\ 
   & 3 & 501 & 87.09 & 13.50 & $\blacktriangle$ \\ 
   & 3 & 508 & 87.07 & \textbf{13.44} & $\blacktriangle$ \\ 
   & 3 & 510 & 87.17 & 14.42 & $\blacktriangle$ \\ 
   & 16 & 1 & 67.48 & 37.66 &  \\ 
\midrule \multirow{7}{*}{\frnd} 
   & 1 & 3 & 81.91 & 18.70 &  \\ 
   & 1 & 8 & 82.55 & 24.45 &  \\ 
   & 2 & 13 & 81.91 & 17.30 &  \\ 
   & 2 & 24 & 85.46 & 17.74 &  \\ 
   & 2 & 51 & 78.72 & 17.17 &  \\ 
   & 3 & 80 & \textbf{85.79} & \textbf{16.72} & $\blacktriangle$ \\ 
   & 16 & 1 & 79.63 & 23.25 &  \\ 
\bottomrule
\end{tabular}

	\end{table}
	
	Now, let's inspect the models corresponding to the minimum mean $\rho$ and highest training accuracy, highlighted in \cref{tbl:CDR:pareto} and inspect the stepwise optimality for those models in \cref{fig:CDR:opt}, again using probability of randomly guessing an optimal move from \cref{sec:opt:rnd} as a benchmark. Note, only one CDR model is plotted for \frnd{10}{10}  as its Pareto front constitutes of only a single model. As one can see for both \jrnd{10}{10} and \jrndn{10}{10}, despite having a higher mean training accuracy overall, the probabilities vary significantly. A lower mean $\rho$ is obtained when the training accuracy is gradually increasing over time, because revisiting \cref{fig:case}, indicates that it's likelier for the resulting makespan to be considerably worse off if suboptimal moves are made at later stages, than at earlier stages. Therefore, it's imperative to make the `best' decision at the `right' moment, not just look at the overall mean performance. Hence, the measure of training accuracy as discussed in \cref{sec:CDR:acc} should take into consideration the impact a suboptimal move yields on a step-by-step basis, e.g. weighted w.r.t. a curve such as depicted in \cref{fig:case}.
	
	\begin{figure}[p]
		\centering
		\includegraphics[width=0.8\linewidth]{figures/exhaust/{trdat.prob.moveIsOptimal.10x10.OPT.equal.best}.pdf}
		\caption{Probability of choosing optimal move for models corresponding to highest training accuracy (grey) and lowest mean deviation from optimality, $\rho$, (black) compared to the baseline of probability of choosing an optimal move at random (dashed).}
		\label{fig:CDR:opt}
	\end{figure}
	
	Let's revert back to the original SDRs discussed in \cref{sec:opt:sdr} and compare the best CDR models, a box-plot for $\rho$ is depicted in \cref{fig:boxplot:CDR}. Firstly, there is a statistical difference between all models, and  clearly the CDR model corresponding to minimum mean $\rho$ value, is the clear winner, and outperforms the  SDRs substantially. However, for \jrnd{10}{10} and \jrndn{10}{10}, where the best model w.r.t. minimum $\rho$ doesn't coincide with the model corresponding to the maximum training accuracy, such as the case with \frnd{10}{10}, then the CDR model shows a lacklustre performance. In some cases it's better off, e.g. compared to LWR, yet doesn't surpass the performance of MWR. This implies, the learning model is overfitting the training data. Results hold for the test set. 
	
	
	\begin{figure}[p]
		\includegraphics[width=1\linewidth]{figures/exhaust/{boxplotRho.CDR.10x10.equal}.pdf}
		\caption{Box plot for deviation from optimality, $\rho$, (\%) for the best CDR models (cf. \cref{tbl:CDR:pareto}) and compared against SDRs from \cref{sec:opt:sdr}, both for training and test sets.}\label{fig:boxplot:CDR}
	\end{figure}
	
	
	\subsection{Interpreting CDR}\label{sec:CDR:interpret}
	\Cref{sec:learningmodels:interpret} showed how to interpret the linear preference  models by their weights. \Cref{fig:CDR:weights}
	depicts the linear weights, $\vec{w}$, from \cref{eq:linear} for all of the CDR models reported in \cref{tbl:CDR:pareto}. The weights have been normalised for clarity purposes, such that it is scaled to $\norm{\vec{w}}=1$, thereby giving each feature their proportional contribution to the  preference $I_j^{CDR}$ defined by \cref{eq:CDR}.
	
	As discussed in \cref{sec:expr:ks} for \jrndn{10}{10}, there is no statistical difference between models 3.501, 3.508 and 3.510 w.r.t. training accuracy. As \cref{fig:CDR:weights} shows, \phimakespan\ and \phiwrmJob\  are similar in value, however it's the third feature that yields the difference in performance. In fact, the contribution from \phiproc\ in 3.501 is on par with \phimac\ in 3.508, as those models are not statistically different w.r.t. $\rho$ performance. However, the decreased contribution of \phimakespan\  in favour for \phimacFree\ in 3.510 results in approximately 1\% increase in $\rho$. Furthermore, it's sufficient to use only \phimakespan\ and \phiwrmJob\  as active features, as model 2.116 has no statistical difference from either 3.508 or 3.510, for both $\rho$ and training accuracy.
	
	Similarly for \jrnd{10}{10}, \phiwrmJob\ and \phiendTime\ are similar for models 3.473 and 3.549, yet statistically significant from one another. There the third feature is the key to the success of the CDR, as opting for \phiwrmMac\ instead of \phiwait\ for 3.549 boosts the $\rho$ performance by about 10\%. 
	
	It's also interesting to inspect the full model for \frnd{10}{10}, 1.16. Despite having similar contributions as all the active features of its best model, 3.80, then the substantial interference from \phijobOps\ along with other features present, hinders the full model from both objectives, i.e., high training accuracy and low $\rho$, thereby stressing the importance of feature selection. 
	
	\begin{figure}
		\centering
		\includegraphics[width=\textwidth]{figures/exhaust/{pareto.equal.phi.j.rnd}.pdf}\\
		\includegraphics[width=\textwidth]{figures/exhaust/{pareto.equal.phi.j.rndn}.pdf}\\
		\includegraphics[width=\textwidth]{figures/exhaust/{pareto.equal.phi.f.rnd}.pdf}
		\caption{Normalised weights for CDR models from \cref{tbl:CDR:pareto}, models are grouped w.r.t. its dimensionality, $d$. Note, a triangle indicates a solution on the Pareto front.}\label{fig:CDR:weights}
	\end{figure}
	
	\subsection{Resampling} \label{sec:sampling} 
	\todo[inline,caption={}]{This is still missing. Sampling strategies that have been applied (but not fully summarised)
	\begin{itemize}
		\item equal probability (current setting)
		\item w.r.t. best and worst case scenario
		\item inverted stepwise optimality
		\item or simply double emphasis on first half vs. second half (and vice versa)	 
	\end{itemize}}
	
	\begin{table}
		\caption{Mean training accuracy and mean expected deviation from optimality, $\rho$, for all CDR models on the Pareto front using various re-sampling probabilities.}\label{tbl:CDR:pareto:all}
		\centering
\begin{tabular}{cr@{.}lllll}\toprule
	Problem  & \multicolumn{2}{c}{PREF} & Sampling & \multicolumn{2}{c}{Accuracy (\%)} & $\rho$ (\%) \\
	  & NrFeat & Model & prob.  & Optimality & Classification &       \\ 
	\midrule \multirow{8}{*}{\jrnd{10}{10}} 
	  & 3      & 486   & wcs    & 90.97      & 62.83          & 12.63 \\ 
	  & 3      & 486   & bcs    & 91.17      & 62.91          & 12.71 \\ 
	  & 3      & 500   & bcs    & 91.20      & 62.89          & 12.78 \\ 
	  & 3      & 556   & equal  & 91.71      & 62.70          & 12.92 \\ 
	  & 3      & 549   & equal  & 91.74      & 62.71          & 12.97 \\ 
	  & 3      & 556   & opt    & 92.03      & 62.47          & 13.51 \\ 
	  & 3      & 498   & wcs    & 92.11      & 62.27          & 19.48 \\ 
	  & 16     & 1     & bcs    & 92.80      & 64.00          & 26.92 \\ 
	\midrule \multirow{25}{*}{\jrndn{10}{10}} 
	  & 3      & 531   & opt    & 85.87      & 59.91          & 12.74 \\ 
	  & 3      & 513   & dbl2nd & 86.67      & 58.41          & 12.80 \\ 
	  & 3      & 520   & wcs    & 86.83      & 58.89          & 13.11 \\ 
	  & 3      & 513   & wcs    & 86.89      & 58.87          & 13.47 \\ 
	  & 3      & 501   & dbl1st & 86.92      & 58.89          & 13.60 \\ 
	  & 3      & 458   & bcs    & 86.99      & 58.83          & 13.72 \\ 
	  & 3      & 544   & opt    & 87.08      & 59.03          & 13.75 \\ 
	  & 3      & 521   & dbl1st & 87.42      & 58.82          & 13.90 \\ 
	  & 3      & 544   & dbl1st & 87.44      & 58.83          & 13.97 \\ 
	  & 3      & 335   & opt    & 87.58      & 58.86          & 14.49 \\ 
	  & 3      & 10    & bcs    & 87.60      & 58.80          & 15.00 \\ 
	  & 3      & 130   & bcs    & 87.70      & 59.10          & 16.37 \\ 
	  & 3      & 378   & wcs    & 87.83      & 59.05          & 18.66 \\ 
	  & 3      & 368   & wcs    & 87.85      & 59.94          & 21.41 \\ 
	  & 3      & 103   & dbl1st & 87.86      & 58.79          & 21.66 \\ 
	  & 3      & 26    & dbl2nd & 88.02      & 58.71          & 23.78 \\ 
	  & 3      & 26    & dbl1st & 88.05      & 58.75          & 23.84 \\ 
	  & 3      & 10    & dbl2nd & 88.09      & 58.67          & 23.90 \\ 
	  & 3      & 94    & dbl2nd & 88.12      & 58.64          & 24.43 \\ 
	  & 3      & 139   & dbl1st & 88.20      & 58.80          & 25.79 \\ 
	  & 3      & 130   & dbl1st & 88.70      & 58.92          & 36.79 \\ 
	  & 3      & 130   & dbl2nd & 88.74      & 58.92          & 36.84 \\ 
	  & 16     & 1     & opt    & 89.07      & 60.04          & 42.07 \\ 
	  & 16     & 1     & dbl1st & 89.19      & 59.78          & 43.35 \\ 
	  & 16     & 1     & dbl2nd & 89.45      & 59.87          & 44.70 \\ 
	\midrule \multirow{5}{*}{\frnd{10}{10}} 
	  & 3      & 260   & opt    & 96.74      & 63.86          & 15.32 \\ 
	  & 3      & 226   & opt    & 96.72      & 63.85          & 15.32 \\ 
	  & 3      & 244   & opt    & 96.76      & 63.80          & 15.61 \\ 
	  & 16     & 1     & wcs    & 96.80      & 71.86          & 21.27 \\ 
	  & 16     & 1     & dbl1st & 96.83      & 70.78          & 22.19 \\ 
	\bottomrule
\end{tabular}
	\end{table}
	
	
	
	%----------------------------- Summarize result
	\section{Conclusions}\label{sec:con}
	Current literature still hold single priority dispatching rules in high regard, as they are simple to implement and quite efficient. 
	However, they are generally taken for granted as there is clear lack of investigation of \emph{how} these dispatching rules actually work, and what makes them so successful (or in some cases unsuccessful)? 
	For instance, of the four SDRs this study focuses on, why does MWR outperform so significantly for job-shop, yet completely fail for flow-shop? 
	MWR seems to be able to adapt to varying distributions of processing times, however manipulating the machine ordering causes MWR to break down. 
	By inspecting optimal schedules, and meticulously researching what's going on, every step of the way of the dispatching sequence, some light is shed where these SDRs vary w.r.t. the problem space at hand. 	\todo{"What general lessons might be learnt from this study?"}
	Once these simple rules are understood, then it's feasible to extrapolate the knowledge gained and create new composite rules that are likely to be successful. 
	
	Creating new dispatching rules is by no means trivial. For job-shop there is the hidden interaction between processing times and machine ordering that's hard to measure.
	Due to this artefact, feature selection is of paramount importance, and then it becomes the case of not having too many features, as they are likely to hinder generalisation due to over-fitting in training. 
	However, the features need to be explanatory enough to maintain predictive ability. 
	For this reason \cref{ch:expr:CDR} was limited to up to three active features, as the full feature set was clearly sub-optimal w.r.t. the SDRs used as a benchmark. 
	By using features based on the SDRs, along with some additional local features describing the current schedule, it was possible to `discover' the SDRs when given only one active feature. %Although there is not much to be gained by these models, they at least are a sanity check the learning models are on the right track. 
	Furthermore, by adding on additional features, a boost in performance was gained, resulting in a composite dispatching rule that outperformed all of the SDR baseline. 
	
	When training the learning model, it's not sufficient to only optimize w.r.t. highest mean training accuracy. As \cref{sec:CDR:pareto} showed, there is a trade-off between making the over-all best decisions versus making the right decision on crucial time points in the scheduling process, as \cref{fig:case} clearly illustrated. It is for this reason, traditional feature selection such as add1 and drop1 were unsuccessful in preliminary experiments, and thus resorting to having to exhaustively search all feature combinations.
	This also opens of the question of how should training accuracy be measured? Since the model is based on learning preferences, both based on optimal versus suboptimal, and then varying degrees of sub-optimality. As we are only looking at the ranks in a black and white fashion, such that the makespans need to be strictly greater to belong to a higher rank, then it can be argued that some ranks should be grouped together if their makespans are sufficiently close. This would simplify the training set, making it (presumably) less of contradictions and more appropriate for linear learning. Or simply the training accuracy could be weighted w.r.t. the  difference in makespan.\todo{Future work topic \#1}
	
	During the dispatching process, there are some pivotal times which need to be especially taken care off. \Cref{fig:case} showed how making suboptimal decisions were more of a factor during the later stages, whereas for flow-shop the case was exact opposite. \todo{Going to wait with this section until \cref{sec:sampling} has been completed}
	
	Despite the abundance of information gathered by following an optimal trajectory, the knowledge obtained is not enough by itself. Since the learning model isn't perfect, it is bound to make a mistake eventually. When it does, the model is in uncharted  territory as there is not certainty the samples already collected are able to explain the current situation. For this we propose investigating features from suboptimal trajectories as well, since the future observations depend on previous predictions. 
	\todo{Future work topic \#2} A straight forward approach would be to inspect the trajectories of promising SDRs or CDRs. 
	In fact, it would be worth while to try out imitation learning by \cite{RossB10,RossGB11}, such that the learned policy following an optimal trajectory is used to collect training data, and the learned model is updated. This can be done over several iterations, with the benefit being, that the states that are likely to occur in practice are investigated, and as such used to dissuade the model from making poor choices. Alas, this comes at great computational cost due to the substantial amounts of states that need to be optimised for their correct labelling. Making it only practical for job-shop of a considerable lower dimension. \todo{Future work topic \#3}
	
	
	Although this study has been structured around the job-shop scheduling problem, it is easily extended to other types of deterministic optimisation problems that involve sequential decision making. \todo{Place work in wider context}	
	The framework presented here collects snap-shots of the state space by following an optimal trajectory, and verifying the resulting optimal makespan from each possible state. 
	From which the stepwise optimality of individual features can be inspected, which could for instance justify omittance in feature selection. \todo{Not done, but possible} 
	Moreover, by looking at the best and worst case scenario of suboptimal dispatches, it is possible to pinpoint vulnerable times in the scheduling process. 
	
	
	
	\bibliographystyle{spmpsci} % spmpsci ??
	\bibliography{../references}  
	
	%\clearpage \listoftodos[Todo and remarks]
	
	
\end{document}



\section{Scheduling Heuristics} \label{sec:constructionjssp}
	
Heuristics algorithms for scheduling are typically either a
construction or improvement heuristics. The improvement heuristic
starts with a complete schedule and then tries to find similar, but
better schedules.  A construction heuristic starts with an empty
schedule and adds one job at a time until the schedule is complete.
The work presented here will focus on construction heuristics,
although the techniques developed could be adapted to improvement
heuristics also.  In scheduling a construction heuristic is typically
implemented as a priority dispatching rule.  These are simple rules
that basically determine which incompleted job should be dispatched
next.  However, knowing which job to dispatch is not sufficient, one
must also know where to place it.  In order to build tight schedules
it would be sensible to place the jobs, once they become available,
such that the machine idle time is minimal.  There may also be a
number of different options for such a placement.
\Cref{fig:jssp:example} illustrates the dispatching process with an
example of a temporal partial schedule for six jobs scheduled on
five-machines.  The numbers in the boxes represent the job's
identification $j$.  The width of the box illustrates the processing
times for a given job for a particular machine $M_a$ (on the vertical
axis).  The dashed boxes represent the resulting partial schedule for
when a particular job is scheduled next.  Moreover, the current
$C_{\max}$ is denoted with a dotted horizontal line.  In the figure one observes
that $J_2$, to be scheduled on $M_3$, could be placed immediately in a
slot between $J_3$ and $J_4$, or after $J_4$.  If
$J_6$ had been scheduled prior a slot would have been created between
it and $J_4$, thus creating a third alternative where 
$J_2$ is placed after $J_6$.  The construction heuristic must therefore decide
where to place the job, and this may be independent of the dispatching
rule applied.  Different placement strategies could be considered, for
example placing a job in smallest feasible slot.  In our preliminary
experiments we have discovered that such a placement could rule out
the possibility of constructing solutions with an optimal makespan.
This problem did not occur when jobs were simply placed as early as feasibly
possible.
	
\begin{figure}[t!]\centering
  \includegraphics[width=0.8\textwidth]{figures/jssp_example_nocolor-eps-converted-to.pdf}
  \caption[Gantt chart of a partial JSP schedule]{Gantt chart of a
    partial JSP schedule after 15 dispatches: Solid and dashed boxes
    represent $\vchi$ and $\mathcal{L}^{(16)}$, respectively. Current
    $C_{\max}$ denoted as dotted line.}
  \label{fig:jssp:example}
\end{figure}
	
	
A \emph{sequence} will refer to the sequential ordering of the
dispatches of tasks to machines, i.e., $(j,a)$; the collective set of
allocated tasks to machines, which is interpreted by its sequence, is
referred to as a \emph{schedule}; a \emph{scheduling policy} will
pertain to the manner in which the sequence is determined.  As shown
in our example given in \Cref{fig:jssp:example}, there are 15
operations already scheduled. The sequence used to create the schedule
was,
\begin{eqnarray}
  \vchi=\left(J_3,J_3,J_3,J_3,J_4,J_4,J_5,J_1,J_1,J_2,J_4,J_6,J_4,J_5,J_3\right)
\end{eqnarray}
hence the current available jobs to be scheduled
$\mathcal{L}=\{J_1,J_2,J_4,J_5,J_6\}$, and will be referred to as our
job-list, describes the 5 potential jobs to be dispatched at step
$k=16$ (note that $J_3$ is completed).
